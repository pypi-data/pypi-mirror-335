\documentclass[a4paper]{article}
\usepackage[normalem]{ulem}
\usepackage[T1]{fontenc}
\usepackage[french]{babel}
\frenchsetup{StandardLayout=true}

\newcommand{\relat}[1]{\textsc{#1}}
\newcommand{\attr}[1]{#1}
\newcommand{\prim}[1]{\uline{#1}}
\newcommand{\foreign}[1]{\#\textsl{#1}}

\title{Conversion en relationnel\\du MCD \emph{inheritance}}
\author{\emph{Généré par Mocodo}}

\begin{document}
\maketitle

\begin{itemize}
  \item \relat{ALIQUET} (\foreign{\prim{magna}}, \foreign{\prim{tellus}})
  \begin{itemize}
    \item Le champ \emph{magna} fait partie de la clé primaire de la table. C'est une clé étrangère qui a migré directement à partir de l'entité \emph{TRISTIS}.
    \item Le champ \emph{tellus} fait partie de la clé primaire de la table. C'est une clé étrangère qui a migré directement à partir de l'entité \emph{DIGNISSIM}.
  \end{itemize}

  \item \relat{CONSEQUAT} (\prim{fermentum}, \attr{dederit})
  \begin{itemize}
    \item Le champ \emph{fermentum} constitue la clé primaire de la table. C'était déjà un identifiant de l'entité \emph{CONSEQUAT}.
    \item Le champ \emph{dederit} était déjà un simple attribut de l'entité \emph{CONSEQUAT}.
  \end{itemize}

  \item \relat{CURABITUR} (\prim{gravida}, \attr{amor})
  \begin{itemize}
    \item Le champ \emph{gravida} constitue la clé primaire de la table. C'était déjà un identifiant de l'entité \emph{CURABITUR}.
    \item Le champ \emph{amor} était déjà un simple attribut de l'entité \emph{CURABITUR}.
  \end{itemize}

  \item \relat{DIGNISSIM} (\prim{tellus}, \attr{terra})
  \begin{itemize}
    \item Le champ \emph{tellus} constitue la clé primaire de la table. C'était déjà un identifiant de l'entité \emph{DIGNISSIM}.
    \item Le champ \emph{terra} était déjà un simple attribut de l'entité \emph{DIGNISSIM}.
  \end{itemize}

  \item \relat{LIBERO} (\prim{posuere}, \attr{lacrima})
  \begin{itemize}
    \item Le champ \emph{posuere} constitue la clé primaire de la table. C'était déjà un identifiant de l'entité \emph{LIBERO}.
    \item Le champ \emph{lacrima} était déjà un simple attribut de l'entité \emph{LIBERO}.
  \end{itemize}

  \item \relat{QUAM} (\prim{cras}, \attr{sed}, \foreign{magna!})
  \begin{itemize}
    \item Le champ \emph{cras} constitue la clé primaire de la table. C'était déjà un identifiant de l'entité \emph{QUAM}.
    \item Le champ \emph{sed} était déjà un simple attribut de l'entité \emph{QUAM}.
    \item Le champ à saisie obligatoire \emph{magna} est une clé étrangère. Il a migré par l'association de dépendance fonctionnelle \emph{VITAE} à partir de l'entité \emph{TRISTIS} en perdant son caractère identifiant.
  \end{itemize}

  \item \relat{SUSCIPIT} (\prim{orci}, \attr{lorem}, \foreign{magna!})
  \begin{itemize}
    \item Le champ \emph{orci} constitue la clé primaire de la table. C'était déjà un identifiant de l'entité \emph{SUSCIPIT}.
    \item Le champ \emph{lorem} était déjà un simple attribut de l'entité \emph{SUSCIPIT}.
    \item Le champ à saisie obligatoire \emph{magna} est une clé étrangère. Il a migré par l'association de dépendance fonctionnelle \emph{RHONCUS} à partir de l'entité \emph{TRISTIS} en perdant son caractère identifiant.
  \end{itemize}

  \item \relat{TRISTIS} (\prim{magna}, \attr{vestibulum}, \foreign{fermentum!}, \attr{type!}, \attr{convallis?}, \attr{ipsum?}, \attr{pulvinar?}, \attr{audis?}, \foreign{gravida?}, \attr{tempor?}, \attr{fugit?})
  \begin{itemize}
    \item Le champ \emph{magna} constitue la clé primaire de la table. C'était déjà un identifiant de l'entité \emph{TRISTIS}.
    \item Le champ \emph{vestibulum} était déjà un simple attribut de l'entité \emph{TRISTIS}.
    \item Le champ à saisie obligatoire \emph{fermentum} est une clé étrangère. Il a migré par l'association de dépendance fonctionnelle \emph{ELIT} à partir de l'entité \emph{CONSEQUAT} en perdant son caractère identifiant.
    \item Un discriminateur à saisie obligatoire \emph{type} est ajouté pour indiquer la nature de la spécialisation. Jamais vide, du fait de la contrainte de totalité.
    \item Le champ à saisie facultative \emph{convallis} a migré à partir de l'entité-fille \emph{SODALES} (supprimée).
    \item Le champ à saisie facultative \emph{ipsum} a migré à partir de l'entité-fille \emph{SODALES} (supprimée).
    \item Le champ à saisie facultative \emph{pulvinar} a migré à partir de l'entité-fille \emph{NEC} (supprimée).
    \item Le champ à saisie facultative \emph{audis} a migré à partir de l'entité-fille \emph{NEC} (supprimée).
    \item Le champ à saisie facultative \emph{gravida} est une clé étrangère. Il a migré à partir de l'entité-fille \emph{NEC} (supprimée) dans laquelle il avait déjà migré à partir de l'entité \emph{CURABITUR}.
    \item Le champ à saisie facultative \emph{tempor} a migré à partir de l'entité-fille \emph{LACUS} (supprimée).
    \item Le champ à saisie facultative \emph{fugit} a migré à partir de l'entité-fille \emph{LACUS} (supprimée).
  \end{itemize}

  \item \relat{ULTRICES} (\foreign{\prim{posuere}}, \foreign{\prim{magna}})
  \begin{itemize}
    \item Le champ \emph{posuere} fait partie de la clé primaire de la table. C'est une clé étrangère qui a migré directement à partir de l'entité \emph{LIBERO}.
    \item Le champ \emph{magna} fait partie de la clé primaire de la table. C'est une clé étrangère qui a migré directement à partir de l'entité \emph{TRISTIS}.
  \end{itemize}

\end{itemize}

\end{document}
